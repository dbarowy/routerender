\documentclass[10pt]{article}

% Lines beginning with the percent sign are comments
% This file has been commented to help you understand more about LaTeX

% DO NOT EDIT THE LINES BETWEEN THE TWO LONG HORIZONTAL LINES

%---------------------------------------------------------------------------------------------------------

% Packages add extra functionality.
\usepackage{times,graphicx,epstopdf,fancyhdr,amsfonts,amsthm,amsmath,algorithm,algorithmic,xspace,hyperref}
\usepackage[left=1in,top=1in,right=1in,bottom=1in]{geometry}
\usepackage{sect sty}	%For centering section headings
\usepackage{enumerate}	%Allows more labeling options for enumerate environments 
\usepackage{epsfig}
\usepackage[space]{grffile}
\usepackage{booktabs}
\usepackage{forest}
\usepackage{array}

% This will set LaTeX to look for figures in the same directory as the .tex file
\graphicspath{.} % The dot means current directory.

\pagestyle{fancy}

\lhead{Final Project}
\rhead{\today}
\lfoot{CSCI 334: Principles of Programming Languages}
\cfoot{\thepage}
\rfoot{Fall 2023}

% Some commands for changing header and footer format
\renewcommand{\headrulewidth}{0.4pt}
\renewcommand{\headwidth}{\textwidth}
\renewcommand{\footrulewidth}{0.4pt}

% These let you use common environments
\newtheorem{claim}{Claim}
\newtheorem{definition}{Definition}
\newtheorem{theorem}{Theorem}
\newtheorem{lemma}{Lemma}
\newtheorem{observation}{Observation}
\newtheorem{question}{Question}

\setlength{\parindent}{0cm}


%---------------------------------------------------------------------------------------------------------

% DON'T CHANGE ANYTHING ABOVE HERE

% Edit below as instructed

\begin{document}
  
\section*{Minimal Project Specification}

Tim Forth, Eric Gage

\subsection{Minimal Grammar}
	$<play> ::=  <defense>$ \\
	$<defense> ::= (<box>, <coverage>)$ \\
	$<box> ::= 34 | 43$ \\
	$<coverage> ::= cover2 $ \\

\smallskip

\subsection{Minimal Semantics}

\begin{table}[h!]
    \begin{tabular}{c|c|c|m{20em}}
        \textbf{Syntax} & \textbf{Abstract Syntax} & \textbf{Type} & \textbf{Meaning} \\
        \hline
        "box" & Box of string & string & tells evaluator what the front 7 looks like i.e. how it should be represented in the svg file; stored as an F\# primitive string data type \\
        \hline
        "coverage" & Coverage of string & string & tells evaluator what the coverage of the defense is i.e. how it should be represented in the svg file; stored as an F\# primitive string data type \\
        \hline
        (box, coverage) & Defense of Box * Coverage & string * string & two-tuple consisting of a box and coverage that tells eval how to draw defense \\
    \end{tabular}
    \caption{Semantics of data types.}
    \label{tab:data}
\end{table}

\end{document}